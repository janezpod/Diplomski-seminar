\documentclass{beamer}

%\setbeameroption{hide notes} % Prosojnice
%\setbeameroption{show only notes} % Zapiski
\setbeameroption{show notes on second screen=right} % Oboje

\usepackage[T1]{fontenc}
\usepackage[utf8]{inputenc}
\usepackage[slovene]{babel}
\usepackage{lmodern}
\usepackage{palatino}
\usefonttheme{serif}
\usepackage{amsmath,amssymb,amsfonts, mathtools}
\graphicspath{{./images/}}

% Shema kodirnika
\usepackage{tikz}
\usetikzlibrary{positioning, fit, calc, automata, arrows}

\usetheme{CambridgeUS}
\usecolortheme{dolphin}
\setbeamertemplate{navigation symbols}{}

\linespread{1}

\theoremstyle{definition}
\newtheorem{definicija}{Definicija}[section]
\newtheorem{izrek}[definicija]{Izrek}
\newtheorem{trditev}[definicija]{Trditev}
\newtheorem{posledica}[definicija]{Posledica}
\newtheorem{opomba}[definicija]{Opomba}

% Absolutna vrednost
\providecommand{\abs}[1]{\left\lvert #1 \right\rvert}

% Naravna števila
\newcommand{\N}{\mathbb{N}}

% Abeceda
\newcommand{\A}{\mathcal{A}}

% Podmnožica KNG
\newcommand{\G}{\mathcal{G}}

\title[Gramatike za stiskanje podatkov]{Kontekstno-neodvisne gramatike \\ za kodiranje in stiskanje podatkov}
\author{Janez Podlogar}
\institute[UL-FMF]{Univerza v Ljubljani, Fakulteta za matematiko in fiziko}
\date[September 2024]{16. 9. 2024}

\begin{document}

\begin{frame}
    \titlepage
    \note{
        Razdeljamo pojme v naslovu.
        \begin{itemize}
            \item Kodiranje je spreminjanje zapisa sporočila.
            \item Stiskanje podatkov je kodiranje, katerega cilj je zapisati sporočilo v zgoščeni
            obliki.
        \end{itemize}
    }
\end{frame}

\section{Kontekstno-neodvisna gramatika}

\begin{frame}
    \frametitle{Kontekstno-neodvisna gramatika}
    \begin{definicija}
        \emph{Kontekstno-neodvisna gramatika} je četverica $G = (V, \Sigma, P, S)$,
        kjer je
        \begin{itemize}
            \item $V$ abeceda \emph{nekončnih simbolov};
            \item $\Sigma$ abeceda \emph{končnih simbolov} taka, da $\Sigma \cap V = \emptyset$;
            \item $P \subseteq V \times ( V \cup \Sigma )^*$ celovita relacija;
            \item $S \in V$ \emph{začetni simbol}.
        \end{itemize}
    \end{definicija}
    \note{
        \begin{itemize}
            \item Abeceda je končna neprazna množica.
            \item Relacija $R \subseteq A \times B$ je \emph{celovita}, če velja 
            $\forall x \in A \ \exists y \in B \colon (x,y) \in R$.
            \item $( V \cup \Sigma )^*$ je množica vseh nizov končne dolžine končnih in nekončnih
            simbolov. Vsebuje tudi prazen niz $\varepsilon$, saj je to niz dolžine $0$.
            \item Prazen niz $\varepsilon$ je enota za operacijo stik.
            \item Okrajšamo z KNG.
        \end{itemize}
        \begin{exampleblock}{Primer}
            Naj bo $ V = \{ S\}$, 
            $\Sigma = \{ a,b,c \}$, 
            $P = \{ (S,\mathit{aSb}), (S , \epsilon) \}$ in
            $S = S$.
        \end{exampleblock}
    }
\end{frame}

\begin{frame}
    \begin{itemize}
        \item Elementom relacije $P$ pravimo \emph{prepisovalna pravila}.
        \item Prepisovalno pravilo $(A, \beta) \in P$ pišemo $A \rightarrow \beta$.
        \item Niz $\alpha A \gamma$ se \emph{prepiše s pravilom} $A \rightarrow \beta$ v 
        $\alpha\beta\gamma$, pišemo $\alpha A \gamma \Rightarrow \alpha\beta\gamma$.
        \item Niz $\alpha $ \emph{izpelje} niz $\beta$, če lahko $\alpha$ prepišemo v $\beta$
        z uporabo končno mnogo prepisovalnih pravil. 
    \end{itemize}
    \pause
    \begin{definicija}
        Jezik KNG je množica nizov, ki jih lahko izpeljemo iz začetnega simbola in 
        ne vsebujejo nekončnih simbolov.
    \end{definicija}
    \note{
        \begin{itemize}
            \item \emph{Leva stran prepisovalnega pravila $A \rightarrow \beta$} je $A$ in 
            \emph{desna stran prepisovalnega pravila} je $\beta$.
        \end{itemize}
        \begin{exampleblock}{Primer}
            prepisovalna pravila zapišemo kot $P = \{ S \rightarrow \mathit{aSb}, S \rightarrow \epsilon \}$.
            Izpeljemo nize
            \[
                S \Rightarrow \epsilon, \ 
                S \Rightarrow \mathit{aSb} \Rightarrow \mathit{ab}, \ 
                S \Rightarrow \mathit{aSb} \Rightarrow \mathit{aaSbb} \Rightarrow \mathit{aabb}, \ 
                \ldots
            \]
            Jezik je $\{ a^n b^n \mid n \geq 0 \}$.
        \end{exampleblock}
    }
\end{frame}

\begin{frame}
    \frametitle{Stiskanje niza}
    \tikzset{block/.style={draw, thick, text width=3cm, minimum height=1.5cm, align=center}, line/.style={-latex}}
    \begin{figure}[H]
        \centering
        \scalebox{0.8}{
        \begin{tikzpicture}
            \node (a) {$w$};
            \node[block,right=of a, xshift=6mm] (b) {prirejanje gramatike};
            \node[right=of b, xshift=3mm] (c) {$G_w$};
            \node[block,right=of c, xshift=3mm] (d) {kodiranje gramatike};
            \node[right=of d, xshift=6mm] (e) {$B(G_w)$};

            \draw[line] (a)-- (b);
            \draw[line] (b)-- (c);
            \draw[line] (c)-- (d);
            \draw[line] (d)-- (e);
        \end{tikzpicture}
        }
    \end{figure}
    \note{
        \begin{itemize}
            \item Glavna ideja je, da je jezik $G_w$ je $\{w\}$, saj lahko enolično rekonstruramo 
            niz $w$.
            \item Alternativni pristop: KNG $G$ je poznana tako kodirniku kot dekodirniku in 
            jezik KNG $G$ vsebuje vse nize, ki jih želimo stistniti. Posamezni niz stisnemo tako,
            da stistnemo izpeljavo niza iz začetnega simbola.
        \end{itemize}
    }
\end{frame}

\subsection{Dopustna gramatika}

\begin{frame}
    \frametitle{Dopustna gramatika}
    \begin{definicija}
        KNG je \emph{deterministična}, če vsak nekončen simbol $A \in V$ nastopa natanko enkrat
        kot leva stran nekega prepisovalnega pravila.
    \end{definicija}
    \pause
    \begin{trditev}
        Jezik deterministične KNG je enojec ali prazna množica.
    \end{trditev}
    \note{
        \begin{itemize}
            \item Prejšnji primer ni deterministična gramatika.
            \item Če odstranimo odstarnimo pravilo $S \rightarrow \mathit{aSb}$,
            postane deterministična, a je jezik $\{ \varepsilon \}$.
            \item Če odstranimo $S \rightarrow \varepsilon$ postane deterministična, a je 
            njen jezik prazna množica.
            \item Determinizem sam po sebi ni dovolj močan, da prepreči praznost jezika.
            Zato bomo od dopustnih gramatik zahtevali, da je njihov jezik neprazen in da ne
            vsebuje pravila oblike $A \rightarrow \varepsilon$.
        \end{itemize}
    }
\end{frame}

\begin{frame}
    \begin{definicija}
        KNG \emph{ne vsebuje neuporabnih simbolov}, 
        če se vsak simbol $ y \in V \cup \Sigma, \ y \neq S $ pojavi vsaj enkrat v izpeljavi 
        niza, ki je v jeziku KNG.
    \end{definicija}
    \note{
        \begin{itemize}
            \item Povedano drugače: simbol je neupraven, če se ne pojavi v nobeni izpeljavi niza,
            ki je v jeziku KNG.
        \end{itemize}
    }
\end{frame}


\begin{frame}
    \begin{definicija}
        KNG je \emph{dopustna gramatika}, če je:
        \begin{itemize}
            \item deterministična,
            \item ne vsebuje neuporabnih simbolov,
            \item ima neprazen jezik,
            \item prazen niz ne nastopa kot desna stran kateregakoli prepisovalnega 
            pravila.
        \end{itemize}
    \end{definicija}
    \pause
    \begin{posledica}
        Jezik dopustne gramatike je enojec. 
    \end{posledica}
\end{frame}

\section{Prirejanj gramatike}

\begin{frame}
    \frametitle{Prirejanje gramatike}
    \tikzset{block/.style={draw, thick, text width=3cm, minimum height=1.5cm, align=center}, line/.style={-latex}}
    \begin{figure}[H]
        \centering
        \scalebox{0.75}{
        \begin{tikzpicture}
            \node (a) {$w$};
            \node[block,right=of a, xshift=6mm] (b) {prirejanje gramatike};
            \node[right=of b, xshift=3mm] (c) {$G_w$};
            \node[block,right=of c, xshift=3mm] (d) {kodiranje gramatike};
            \node[right=of d, xshift=6mm] (e) {$B(G_w)$};

            \draw[line] (a)-- (b);
            \draw[line] (b)-- (c);
            \draw[line] (c)-- (d);
            \draw[line] (d)-- (e);
        \end{tikzpicture}
        }
    \end{figure}
    Od tu naprej naj bo:
    \begin{itemize}
        \item $\A$ poljubna abeceda , $\abs{\A} \geq 2$;
        \item $\{ A_0, A_1, \ldots \}$ končna množica ,
        $\A \cap \{ A_0, A_1, \ldots \} = \emptyset$.
    \end{itemize}
\end{frame}

\begin{frame}
    \begin{definicija}
        Naj bo $\G(\A)$ množica vseh KNG $G$, ki zadostujejo:
        \begin{enumerate}
            \item $G$ je dopustna gramatika;
            \item $\Sigma \subseteq \A$;
            \item $V = \{ A_0, A_1, \ldots, A_{\abs{V}-1} \}$;
            \item $S=A_0$;
            \item Če naštejemo nekončne simbole $V$ v vrstnem redu prve pojavitve pri izpeljavi
            niza gramatike, dobimo zaporedje $A_0, A_1, A_2, \ldots, A_{\abs{V}-1}$.
        \end{enumerate}
    \end{definicija}
    \note{
    \begin{itemize}
        \item Z zahtevo $5.$ so nekončni simboli poimenovani po edinstvem vrstnem redu.
        Ta vrstni red omogoča pravilno dekodiranje.
        \item KNG $G \notin \G(\A)$, ki izpolnjuje zahtevi $1.$ in $2.$, preimenujmo nekončne
        simbole tako, da izpolnimo $5.$ zahtevo. Potem izpolnimo tudi $3.$ in $4.$ zahtevi. 
        Dobimo $[G] \in \G(\A)$, ki jo imenujemo \emph{kanonično oblika $G$}, in velja da sta jezika
        $[G]$ in $G$ enaka.
    \end{itemize}
}
\end{frame}

\begin{frame}
    \tikzset{block/.style={draw, thick, text width=3cm, minimum height=1.5cm, align=center}, line/.style={-latex}}
    \begin{figure}[H]
        \centering
        \scalebox{0.8}{
        \begin{tikzpicture}
            \node (a) {$w$};
            \node[block,right=of a, xshift=6mm] (b) {prirejanje gramatike};
            \node[right=of b, xshift=3mm] (c) {$G_w$};
            \node[block,right=of c, xshift=3mm] (d) {kodiranje gramatike};
            \node[right=of d, xshift=6mm] (e) {$B(G_w)$};

            \draw[line] (a)-- (b);
            \draw[line] (b)-- (c);
            \draw[line] (c)-- (d);
            \draw[line] (d)-- (e);
        \end{tikzpicture}
        }
    \end{figure}
    \begin{definicija}
        \emph{Prirejanje gramatike nizu abecede $\A$} je preslikava
        \begin{align*}
            \pi \colon \A^+ &\to \G(\A), \\
            w &\mapsto G_w.
        \end{align*}
    \end{definicija}
\end{frame}

\begin{frame}
    \begin{definicija}
        Z $\G^*(\A)$ označimo pravo podmnožico množice $\G(\A)$, da za vsak $G \in \G^*(\A)$ velja
        \[
            \forall A,B \in V, \ A \neq B \colon f_G^\infty(A) \neq f_G^\infty(B).
        \]
    \end{definicija}
    \pause
    \begin{definicija}
        Z $\abs{G}$ označimo vsoto dolžin desnih strani prepisovalnih pravil KNG $G$.
    \end{definicija}
    \note{
        \begin{itemize}
            \item Za prirejanja, ki jih predstavimo potrebujemo še eno omejitev s 
            katero se v predstavitvi ne obremenjujemo.
        \end{itemize}
    }
\end{frame}

\subsection{Asimptotsko kompaktno prirejanje gramatike}

\begin{frame}
    \frametitle{Asimptotsko kompaktno prirejanje gramatike}
    \begin{definicija}
        Prirejanje gramatike nizu abecede $\A$ je \textit{asimptotsko kompaktno}, če za vsak niz
        $w \in \A^+$ velja $G_w \in \G^*(\A)$ in je
        \[
            \lim_{n \rightarrow \infty} \max_{w \in \A^n} \frac{\abs{G_w}}{\abs{w}} = 0.
        \]
    \end{definicija}
    \note{
        \begin{itemize}
            \item Primer je bisekcijsko prirejanje gramatike.
        \end{itemize}
    }
\end{frame}

\subsection{Neskrčljivo prirejanje gramatike}

\begin{frame}
    \frametitle{Neskrčljivo prirejanje gramatike}
    \begin{definicija}
        Pravimo, da je $G \in \G^*(\A)$ \emph{neskrčljiva gramatika}, če:
        \begin{enumerate}
            \item Vsak $A \in V$, $A \neq S$ nastopa vsaj dvakrat v desni strani prepisovalnih pravil;
            \item Ne obstajata $y_1,y_2 \in V \cup \Sigma$, da niz $y_1y_2$ nastopa kot podniz desne 
            strani kateregali prepisovalnega pravila več kot enkrat na neprekrivajočih se mestih. 
        \end{enumerate}
    \end{definicija}
    \pause
    \begin{definicija}
        \emph{Neskrčljivo prirejanje gramatike nizu abecede $\A$} vsakemu nizu abecede $\A$ priredi
        neskrčljivo gramatiko.
    \end{definicija}
    \note{
        \begin{itemize}
            \item Primer prekrivajočih mest je $\underline{11}1$ in  $1\underline{11}$.
            \item Primer neprekriavjočih mest je $\underline{11} \ \underline{11}$.
            \item Različna neskrčjiva prirejanja gramatike dobimo tako, da izvajamo 
            različne nabore skrčitvenih pravil.
            \item Primer je metodo najdaljšega ujemajočega podniza.
        \end{itemize}
    }
\end{frame}

\section{Binarno kodiranje gramatike}

\begin{frame}
    \frametitle{Binarno kodiranje gramatike}
    \tikzset{block/.style={draw, thick, text width=3cm, minimum height=1.5cm, align=center}, line/.style={-latex}}
    \begin{figure}[H]
        \centering
        \scalebox{0.8}{
        \begin{tikzpicture}
            \node (a) {$w$};
            \node[block,right=of a, xshift=6mm] (b) {prirejanje gramatike};
            \node[right=of b, xshift=3mm] (c) {$G_w$};
            \node[block,right=of c, xshift=3mm] (d) {kodiranje gramatike};
            \node[right=of d, xshift=6mm] (e) {$B(G_w)$};

            \draw[line] (a)-- (b);
            \draw[line] (b)-- (c);
            \draw[line] (c)-- (d);
            \draw[line] (d)-- (e);
        \end{tikzpicture}
        }
    \end{figure}
    \begin{definicija}
        \emph{Binarno kodiranje dopustne gramatike} je preslikava 
        \begin{align*}
            B \colon \G(\A) &\to \{ 0, 1 \}^+, \\
            G &\mapsto B(G).
        \end{align*}
    \end{definicija} 
\end{frame}

\begin{frame}
    \begin{izrek}
        Obstaja bijektivno binarno kodiranje dopustne gramatike, da
        \begin{enumerate}
            \item $ \forall G_1, G_2 \in \G(\A), G_1 \neq G_2 $ niz $B(G_1)$ ni predpona niza $B(G_2)$,
            \item $ \forall G \in \G(\A) \colon \abs{B(G)} \leq \abs{\A} + 4 \abs{G} + \lceil H(G) \rceil$.
        \end{enumerate}
    \end{izrek}
    \note{
        \begin{itemize}
            \item Abeceda je poznana tako kodirniku kot dekodirniku.
        \end{itemize}
    }
\end{frame}

\begin{frame}
    \begin{definicija}
        \emph{Stiskanje niza abecede $\A$ z gramatikami $\G(\A)$} je 
        par preslikav kodne in dekodne preslikave $\varPhi = (\kappa, \delta)$. Kodna preslikava je
        \begin{align*}
            \kappa \colon \A^+ &\to \{ 0, 1\}^+,\\
            w &\mapsto B(\pi(w)),
        \end{align*}
        kjer je $\pi$ prirejanje gramatike nizu abecede $\A$ in
        $B$ binarno kodiranje dopustne gramatike.
    \end{definicija}
    \note{
        \begin{itemize}
            \item Odvečnost meri količino ponavljajočih se ali predvidljivih podatkov znotraj
            sporočila, ki jih je mogoče odstraniti, da se prihrani prostor, brez izgube informacije.
            \item Odvečnost stiskanja z asimptotsko kompaktnim prirejanjem konvergira proti $0$ v 
            odvisnosti od izbire kodiranja znotraj razreda.
            \item Stiskanje z neskrčljivim prirejanjem tudi stiskanje z asimptotsko kompaktnim 
            prirejanjem in odvečnost konvergira enakomerno proti $0$ za vsa stiskanja z neskrčljivim 
            prirejanjem, vsaj tako hitro kot $\frac{\log_2 \log_2(n)}{\log_2(n)}$ pomnoženo z neko konstanto.
        \end{itemize}
    }
\end{frame}


\end{document}