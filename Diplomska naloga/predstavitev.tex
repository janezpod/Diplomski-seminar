\documentclass{beamer}

%\setbeameroption{hide notes} % Prosojnice
%\setbeameroption{show only notes} % Zapiski
\setbeameroption{show notes on second screen=right} % Oboje

\usepackage[T1]{fontenc}
\usepackage[utf8]{inputenc}
\usepackage[slovene]{babel}
\usepackage{lmodern}
\usepackage{palatino}
\usefonttheme{serif}
\usepackage{amsmath,amssymb,amsfonts, mathtools}
\graphicspath{{./images/}}

% Shema kodirnika
\usepackage{tikz}
\usetikzlibrary{positioning, fit, calc, automata, arrows}

\usetheme{CambridgeUS}
\usecolortheme{dolphin}
\setbeamertemplate{navigation symbols}{}

\linespread{1}

\theoremstyle{definition}
\newtheorem{definicija}{Definicija}[section]
\newtheorem{izrek}[definicija]{Izrek}
\newtheorem{trditev}[definicija]{Trditev}
\newtheorem{posledica}[definicija]{Posledica}
\newtheorem{opomba}[definicija]{Opomba}

% Absolutna vrednost
\providecommand{\abs}[1]{\left\lvert #1 \right\rvert}

% Naravna števila
\newcommand{\N}{\mathbb{N}}

% Abeceda
\newcommand{\A}{\mathcal{A}}

% Podmnožica KNG
\newcommand{\G}{\mathcal{G}}

\title[Gramatike za stiskanje podatkov]{Kontekstno-neodvisne gramatike \\ za kodiranje in stiskanje podatkov}
\author{Janez Podlogar}
\institute[UL-FMF]{Univerza v Ljubljani, Fakulteta za matematiko in fiziko}
\date[September 2024]{16. 9. 2024}

\begin{document}

\begin{frame}
    \titlepage
    \note{
        Razdeljamo pojme v naslovu.
        \begin{itemize}
            \item Kodiranje je spreminjanje zapisa sporočila.
            \item Stiskanje podatkov je kodiranje, katerega cilj je zapisati sporočilo v zgoščeni
            obliki.
        \end{itemize}
    }
\end{frame}

\section{Kontekstno-neodvisna gramatika}

\begin{frame}
    \frametitle{Kontekstno-neodvisna gramatika}
    \begin{definicija}
        \emph{Kontekstno-neodvisna gramatika} je četverica $G = (V, \Sigma, P, S)$,
        kjer je
        \begin{itemize}
            \item $V$ abeceda \emph{nekončnih simbolov};
            \item $\Sigma$ abeceda \emph{končnih simbolov} taka, da $\Sigma \cap V = \emptyset$;
            \item $P \subseteq V \times ( V \cup \Sigma )^*$ celovita relacija;
            \item $S \in V$ \emph{začetni simbol}.
        \end{itemize}
    \end{definicija}
    \note{
        \begin{itemize}
            \item Abeceda je končna neprazna množica.
            \item Relacija $R \subseteq A \times B$ je \emph{celovita}, če velja 
            $\forall x \in A \ \exists y \in B \colon (x,y) \in R$.
            \item Kleenijevo zaprtje abecede $V$ je $V^*$, ki vsebuje prazen niz $\varepsilon$
            in je zaprta za operacijo stik. Torej vsebuje prazen niz $\varepsilon$ in vse nize
            končne dolžine, ki jih lahko tvorimo z stikanjem poljubnih elemntov iz $V$.
            \item Prazen niz $\varepsilon$ je enota za operacijo stik.
            \item Standardno z velikimi tiskanimi črkami označujemo nekončne simbole, z malimi 
            tiskanimi črkami označujemo končne simbole in z grškimi črkami označujemo končne nize
            nekončnih in končnih simbolov. Ko govorimo o poljubnem simbolu, ga označimo z $y$.
        \end{itemize}
    }
\end{frame}

\begin{frame}
    \begin{itemize}
        \item Elementom relacije $P$ pravimo \emph{prepisovalna pravila}.
        \item Prepisovalno pravilo $(A, \beta) \in P$ pišemo $A \rightarrow \beta$.
        \item Niz $\alpha A \gamma$ se \emph{prepiše s pravilom} $A \rightarrow \beta$ v 
        $\alpha\beta\gamma$, pišemo $\alpha A \gamma \Rightarrow \alpha\beta\gamma$.
        \item Niz $\alpha $ \emph{izpelje} $\beta$, če lahko $\alpha$ prepišemo v $\beta$
        z uporabo končno mnogo prepisovalnih pravil. 
    \end{itemize}
    \pause
    \begin{definicija}
        Jezik KNG $G$ je množica nizov $L(G)$, ki jih lahko izpeljemo iz začetnega simbola in 
        ne vsebujejo nekončnih simbolov.
    \end{definicija}
    \note{
        \begin{itemize}
            \item Za kontekstno-neodvisno gramatiko uporabljamo okrajšavo KNG in se sklicujemo
            na komponente četrverice $(V, \Sigma, P, S)$ brez njene eksplicitne navedbe.
            \item \emph{Leva stran prepisovalnega pravila $A \rightarrow \beta$} je $A$ in 
            \emph{desna stran prepisovalnega pravila} je $\beta$.
        \end{itemize}
        Napišemo primer in preverimo, da je res KNG.
        \begin{exampleblock}{Primer}
            Naj bo $ V = \{ S\}$, 
            $\Sigma = \{ a,b,c \}$, 
            $P = \{ S \rightarrow \mathit{aSb}, S \rightarrow \epsilon \}$ in
            $S = S$.
            Izpeljemo nize
            \[
                S \Rightarrow \epsilon, \ 
                S \Rightarrow \mathit{aSb} \Rightarrow \mathit{ab}, \ 
                S \Rightarrow \mathit{aSb} \Rightarrow \mathit{aaSbb} \Rightarrow \mathit{aabb}, \ 
                \ldots
            \]
            Jezik je $\{ a^n b^n \mid n \geq 0 \}$.
        \end{exampleblock}
    }
\end{frame}

\begin{frame}
    \frametitle{Stiskanje niza}
    \tikzset{block/.style={draw, thick, text width=3cm, minimum height=1.5cm, align=center}, line/.style={-latex}}
    \begin{figure}[H]
        \centering
        \scalebox{0.8}{
        \begin{tikzpicture}
            \node (a) {$w$};
            \node[block,right=of a, xshift=6mm] (b) {prirejanje gramatike};
            \node[right=of b, xshift=3mm] (c) {$G_w$};
            \node[block,right=of c, xshift=3mm] (d) {kodiranje gramatike};
            \node[right=of d, xshift=6mm] (e) {$B(G_w)$};

            \draw[line] (a)-- (b);
            \draw[line] (b)-- (c);
            \draw[line] (c)-- (d);
            \draw[line] (d)-- (e);
        \end{tikzpicture}
        }
    \end{figure}
    \note{
        \begin{itemize}
            \item Glavna ideja je, da je jezik $G_w$ je $\{w\}$, saj lahko enolično rekonstruramo 
            niz $w$.
            \item Alternativni pristop je: KNG $G$ je poznana tako kodirniku kot dekodirniku in 
            jezik KNG $G$ vsebuje vse nize, ki jih želimo stistniti. Posamezni niz stisnemo tako,
            da stistnemo izpeljavo niza iz začetnega simbola.
        \end{itemize}
    }
\end{frame}

\subsection{Dopustna gramatika}

\begin{frame}
    \frametitle{Dopustna gramatika}
    \begin{definicija}
        KNG je \emph{deterministična}, če vsak nekončen simbol $A \in V$ nastopa natanko enkrat
        kot leva stran nekega prepisovalnega pravila. Sicer je 
        \emph{nedeterministična}.
    \end{definicija}
    \pause
    \begin{trditev}
        Jezik deterministične KNG je enojec ali prazna množica.
    \end{trditev}
    \note{
        \begin{itemize}
            \item Prejšnji primer ni deterministična gramatika. Kaj se zgodi, če odstranimo
            $S \rightarrow \varepsilon$, da postane deterministična.
            \item Determinizem sam po sebi ni dovolj močan, da prepreči praznost jezika, kot
            pokažeta sledeča primera, kjer se v izpeljavi “zaciklamo”. Zato bomo od dopustnih
            gramatik zahtevali, da je njihov jezik neprazen.
            \item Prav tako nočemo, da je ta enojec prazen niz $\varepsilon$. To se zgodi, če
            odstarnimo pravilo $S \rightarrow \mathit{aSb}$.
        \end{itemize}
    }
\end{frame}

\begin{frame}
    \begin{definicija}
        KNG $G$ \emph{ne vsebuje neuporabnih simbolov}, 
        če se vsak simbol $ y \in V \cup \Sigma, \ y \neq S $ pojavi vsaj enkrat v izpeljavi 
        niza, ki je v jeziku KNG $G$.
    \end{definicija}
    \note{
        Povedano drugače:
        \begin{itemize}
            \item Simbol je odvečen, če se ne pojavi v nobeni izpeljavi niza, ki je v jeziku KNG.
        \end{itemize}
    }
\end{frame}


\begin{frame}
    \begin{definicija}
        KNG $G$ je \emph{dopustna gramatika}, če je:
        \begin{itemize}
            \item deterministična,
            \item ne vsebuje neuporabnih simbolov,
            \item ima neprazen jezik,
            \item prazen niz ne nastopa kot desna stran kateregakoli prepisovalnega 
            pravila.
        \end{itemize}
    \end{definicija}
    \pause
    \begin{posledica}
        Jezik dopustne gramatike je enojec. 
    \end{posledica}
    \note{
        Vse povedano sestavimo skupaj.
    }
\end{frame}

\section{Prirejanj gramatike}

\begin{frame}
    \frametitle{Prirejanje gramatike}
    \tikzset{block/.style={draw, thick, text width=3cm, minimum height=1.5cm, align=center}, line/.style={-latex}}
    \begin{figure}[H]
        \centering
        \scalebox{0.75}{
        \begin{tikzpicture}
            \node (a) {$w$};
            \node[block,right=of a, xshift=6mm] (b) {prirejanje gramatike};
            \node[right=of b, xshift=3mm] (c) {$G_w$};
            \node[block,right=of c, xshift=3mm] (d) {kodiranje gramatike};
            \node[right=of d, xshift=6mm] (e) {$B(G_w)$};

            \draw[line] (a)-- (b);
            \draw[line] (b)-- (c);
            \draw[line] (c)-- (d);
            \draw[line] (d)-- (e);
        \end{tikzpicture}
        }
    \end{figure}
    Od tu naprej naj bo:
    \begin{itemize}
        \item $\A$ poljubna abeceda , $\abs{\A} \geq 2$;
        \item $\{ A_0, A_1, \ldots \}$ končna množica ,
        $\A \cap \{ A_0, A_1, \ldots \} = \emptyset$.
    \end{itemize}
\end{frame}

\begin{frame}
    \begin{definicija}
        Naj bo $\G(\A)$ množica vseh KNG $G$, ki zadostujejo:
        \begin{enumerate}
            \item $G$ je dopustna gramatika;
            \item $\Sigma \subseteq \A$;
            \item $V = \{ A_0, A_1, \ldots, A_{\abs{V}-1} \}$;
            \item $S=A_0$;
            \item Če naštejemo nekončne simbole $V$ v vrstnem redu prve pojavitve pri izpeljavi
            niza gramatike, dobimo zaporedje $A_0, A_1, A_2, \ldots, A_{\abs{V}-1}$.
        \end{enumerate}
    \end{definicija}
    \note{
    \begin{itemize}
        \item Z zahtevo $5.$ so nekončni simboli poimenovani po edinstvem vrstnem redu.
        Ta vrstni red bo dekodirniku omogočil pravilno določiti naslednji nekončni simbol. Opomni
        \item KNG $G \notin \G(\A)$, ki izpolnjuje zahtevi $1.$ in $2.$, preimenujmo nekončne
        simbole tako, da izpolnimo $5.$ zahtevo. Potem izpolnimo tudi $3.$ in $4.$ zahtevi. 
        Dobimo $[G] \in \G(\A)$, ki jo imenujemo \emph{kanonično oblika $G$}, in velja da sta jezika
        $[G]$ in $G$ enaka.
    \end{itemize}
}
\end{frame}

\begin{frame}
    \tikzset{block/.style={draw, thick, text width=3cm, minimum height=1.5cm, align=center}, line/.style={-latex}}
    \begin{figure}[H]
        \centering
        \scalebox{0.8}{
        \begin{tikzpicture}
            \node (a) {$w$};
            \node[block,right=of a, xshift=6mm] (b) {prirejanje gramatike};
            \node[right=of b, xshift=3mm] (c) {$G_w$};
            \node[block,right=of c, xshift=3mm] (d) {kodiranje gramatike};
            \node[right=of d, xshift=6mm] (e) {$B(G_w)$};

            \draw[line] (a)-- (b);
            \draw[line] (b)-- (c);
            \draw[line] (c)-- (d);
            \draw[line] (d)-- (e);
        \end{tikzpicture}
        }
    \end{figure}
    \begin{definicija}
        \emph{Prirejanje gramatike nizu abecede $\A$} je preslikava
        \begin{align*}
            \pi \colon \A^+ &\to \G(\A), \\
            w &\mapsto G_w.
        \end{align*}
    \end{definicija}
    \note{
        \begin{itemize}
            \item Sedaj lahko definiramo prirejanje gramatike nizu abecede.
        \end{itemize}
    }
\end{frame}

\begin{frame}
    \begin{definicija}
        Z $\G^*(\A)$ označimo pravo podmnožico množice $\G(\A)$, da za vsak $G \in \G^*(\A)$ velja
        \[
            \forall A,B \in V, \ A \neq B \colon f_G^\infty(A) \neq f_G^\infty(B).
        \]
    \end{definicija}
    \pause
    \begin{definicija}
        Z $\abs{G}$ označimo vsoto dolžin desnih strani prepisovalnih pravil KNG $G$.
    \end{definicija}
    \note{
        \begin{itemize}
            \item Sedaj lahko definiramo prirejanje gramatike nizu abecede.
            \item Za prirejanja, ki jih predstavimo potrebujemo še eno omejitev s 
            katero se v predstavitvi ne obremenjujemo.
            \item Definicijo pa rabimo razumeti.
        \end{itemize}
    }
\end{frame}

\subsection{Asimptotsko kompaktno prirejanje gramatike}

\begin{frame}
    \frametitle{Asimptotsko kompaktno prirejanje gramatike}
    \begin{definicija}
        Prirejanje gramatike nizu abecede $\A$ je \textit{asimptotsko kompaktno}, če za vsak niz
        $w \in \A^+$ velja $G_w \in \G^*(\A)$ in je
        \[
            \lim_{n \rightarrow \infty} \max_{w \in \A^n} \frac{\abs{G_w}}{\abs{w}} = 0.
        \]
    \end{definicija}
\end{frame}

\begin{frame}
    \frametitle{Bisekcijsko prirejanje gramatike}
    Naj bo $w \in \A^+$, $w_1w_2 \cdots w_n$ predstavitev niza $w$ s črkami abecede.
    \begin{block}{Bisekcijska členitev niza}
        $\sigma_{\text{bis}}(w) = \{ w \} \cup \Bigl\{ w_iw_{i+1} \cdots w_j \Bigm| \log_2(j-i-1) \in \N_0
        \text{ in } \frac{i-1}{j-i-1} \in \N_0 \bigr\}$.
    \end{block}
    \pause
    \begin{block}{Končni simboli}
        $\Sigma = \{ u \in \sigma_{\text{bis}}(w) \mid \abs{u} = 1 \}$.
    \end{block}
    \begin{block}{Nekončni simboli}
        $V = \{ A_u \mid u \in \sigma_{\text{bis}}(w) \}$.
    \end{block}
    \begin{block}{Začetni simbol}
        $S = A_w$.
    \end{block}
    \note{
        \begin{itemize}
            \item Prvi pogoj pove, da $\sigma_{\text{bis}}(w)$ vsebuje le podnize dolžine $2^n$ 
            za $n \in \N_0$.  Drugi pogoj pove, da se po nizu $w$ ``premikamo'' s korakom dolžine podniza.
        \end{itemize}
        \begin{exampleblock}{Primer}
            Naj bo $w = 0001010$. Poiščimo podmnožico nizov $\sigma_{\text{bis}}(w)$. Po definiciji je 
            $0001010 \in \sigma_{\text{bis}}(w)$. Ker je $\abs{w} = 7$, zaradi prvega pogoja gledamo 
            podnize dolžine $1, 2, 4$. Vsi nizi dolžine $1$ so trivialno vsebovani. Poglejmo podnize 
            dolžine $2$. Zaradi drugega pogoja se po nizu ``premikamo'' s korakom dolžine $2$.
            Nato se po nizu ``premikamo'' s korakom dolžine $4$. 
            Podčratmo vse nize, ki so vsebovani v $\sigma_{\text{bis}}(w)$. Vidimo zakaj jo imenujemo
            \emph{bisekcijska} dopustna gramatika:
            \begin{gather*}
                \underline{\underline{\underline{\underline{0} \, \underline{0}} \,
                \underline{\underline{0} \, \underline{1}}} \, \underline{\underline{0} \, \underline{1}}
                \, \underline{0}}.
            \end{gather*}
        \end{exampleblock}
    }
\end{frame}

\begin{frame}
    \begin{block}{Prepisovalna pravila}
        Za vsak $u \in \sigma_{\text{bis}}(w)$ velja:
        \begin{itemize}
            \item Če je $\abs{u} = 1$, je prepisovalno pravilo oblike
            \[
                A_u \rightarrow u.
            \]
            \item Če je $\log_2(\abs{u}) \in \N$, je prepisovalno pravilo oblike 
            \[
                A_u \rightarrow A_lA_d,
            \]
            kjer je $\abs{l} = \abs{d}$ in $ld = u$.
            \item Sicer je $\log_2(\abs{u}) \notin \N$. Prepisovalno pravilo je oblike 
            \[
                A_u \rightarrow A_{u_1}A_{u_2} \cdots A_{u_m},
            \]
            kjer je  $\abs{w} > \abs{u_1} > \abs{u_2} > \cdots > \abs{u_m}$ in $u_1u_2 \cdots u_m = w$
        \end{itemize}
    \end{block}
    \note{
        \begin{itemize}
            \item Niz $u$ zapišemo kot stik dveh enako dolgih nizov $l$ in $d$.
            \item Prej smo izračunali $\sigma_{\text{bis}}(w) = \{ 0001010, 0001, 01, 00, 1, 0 \}$ 
            in iz podčranega niza vidimo, da je $\underline{0001} \, \underline{01} \, \underline{0}$
            iskano razbitje niza $w$. Prepisovalna pravila dopustne gramatike $G^\text{bis}_w$ so
            \begin{align*}
                A_w &\rightarrow A_{0001}A_{01}A_{0}, \\
                A_{0001} &\rightarrow A_{00}A_{01}, \\
                A_{01} &\rightarrow A_{0}A_{1}, \\
                A_{00} &\rightarrow A_{0}A_{0}, \\
                A_{1} &\rightarrow 1, \\
                A_{0} &\rightarrow 0.
            \end{align*}
            \item Seveda moramo še ustrezno preimenovati nekončne simbole, da dobimo kanonično
            obliko.
        \end{itemize}
    }
\end{frame}

\subsection{Neskrčljivo prirejanje gramatike}

\begin{frame}
    \frametitle{Neskrčljivo prirejanje gramatike}
    \begin{definicija}
        Pravimo, da je $G \in \G^*(\A)$ \emph{neskrčljiva gramatika}, če:
        \begin{enumerate}
            \item Vsak $A \in V$, $A \neq S$ nastopa vsaj dvakrat v desni strani prepisovalnih pravil;
            \item Ne obstajata $y_1,y_2 \in V \cup \Sigma$, da niz $y_1y_2$ nastopa kot podniz desne 
            strani kateregali prepisovalnega pravila več kot enkrat na neprekrivajočih se mestih. 
        \end{enumerate}
    \end{definicija}
    \pause
    \begin{definicija}
        \emph{Neskrčljivo prirejanje gramatike nizu abecede $\A$} vsakemu nizu abecede $\A$ priredi
        neskrčljivo gramatiko.
    \end{definicija}
    \note{
        \begin{itemize}
            \item Primer prekrivajočih mest je $\underline{11}1$ in  $1\underline{11}$.
            \item Različna neskrčjiva prirejanja gramatike dobimo tako, da izvajamo 
            različne sisteme ``skrčitev''.
            \item Poglejmo si en tak sistem in sicer metodo najdaljšega ujemajočega podniza.
            \item V delu je predstavljenih več pravil, saj lahko z njimi oblikujemo različne
            sisteme redukcij.
        \end{itemize}
    }
\end{frame}

\begin{frame}
    \frametitle{Metoda najdaljšega ujemajočega podniza}
    Naj bo $w \in \A$. Začnemo s trivilano dopustno gramatiko
    \[
    S \rightarrow w.
    \]
    Ponavljaj dokler lahko: poiščemo najdaljši podniz, ki se pojavi dvakrat na neprekriavjočih 
    se mestih levih  strani prepisovalnih pravil, uporabi eno izmed skrčitvenih pravil:
\end{frame}

\begin{frame}
    \frametitle{Metoda najdaljšega ujemajočega podniza}
    \begin{block}{Skrčitveno pravilo 1}
        Če obstaja prepisovalno pravilo
        \[
            A \rightarrow \alpha_1 \beta \alpha_2 \beta \alpha_3,
        \]
        kjer so $\alpha_1, \alpha_2, \alpha_3, \beta \in (V \cup \Sigma)^*$ in $\abs{\beta} \geq 2$,
        izberemo $B \notin V \cup \Sigma$ in
        \begin{itemize}
            \item dodamo pravilo $B \rightarrow \beta$;
            \item nadomestimo pravilo $A \rightarrow \alpha_1 \beta \alpha_2 \beta \alpha_3$ z 
            $A \rightarrow \alpha_1 B \alpha_2 B \alpha_3$.
        \end{itemize}
    \end{block}
    \note{
        \begin{itemize}
            \item Jezik spremenjene dopsutne gramatike je enak jeziku stare.
            \item Nova dopustna gramatika je ``bližje'' zadostitvi $2.$ zahtevi 
            definicije neskrčljive gramatike:

            Ne obstajata $y_1,y_2 \in V \cup \Sigma$, da niz $y_1y_2$ nastopa kot podniz desne 
            strani kateregali prepisovalnega pravila več kot enkrat na neprekrivajočih se mestih.
        \end{itemize}
    }
\end{frame}

\begin{frame}
    \frametitle{Metoda najdaljšega ujemajočega podniza}
    \begin{block}{Skrčitveno pravilo 2}
        Če obstajata prepisovalni pravili
        \begin{align*}
            A &\rightarrow \alpha_1 \beta \alpha_2, \\
            B &\rightarrow \alpha_3 \beta \alpha_4,
        \end{align*}
        kjer so $\alpha_1, \alpha_2, \alpha_3, \alpha_4, \beta \in (V \cup \Sigma)^*$, $\abs{\beta} \geq 2$,
        $(\alpha_1 \neq \varepsilon \vee \alpha_2 \neq \varepsilon)$ in 
        $(\alpha_3 \neq \varepsilon \vee \alpha_4 \neq \varepsilon)$,
        izberemo $C \notin V \cup \Sigma$ in
        \begin{itemize}
            \item dodamo pravilo $C \rightarrow \beta$;
            \item nadomestimo pravilo $A \rightarrow \alpha_1 \beta \alpha_2$ 
            z $A \rightarrow \alpha_1 C \alpha_2$;
            \item nadomestimo pravilo $A \rightarrow \alpha_3 \beta \alpha_4$ 
            z $A \rightarrow \alpha_3 C \alpha_4$.
        \end{itemize}
    \end{block}
    \note{
        \begin{itemize}
            \item Jezik spremenjene dopsutne gramatike je enak jeziku stare.
            \item Nova dopustna gramatika je ``bližje'' zadostitvi $2.$ zahtevi 
            definicije neskrčljive gramatike:

            Ne obstajata $y_1,y_2 \in V \cup \Sigma$, da niz $y_1y_2$ nastopa kot podniz desne 
            strani kateregali prepisovalnega pravila več kot enkrat na neprekrivajočih se mestih.
        \end{itemize}
    }
\end{frame}

\begin{frame}
    \frametitle{Metoda najdaljšega ujemajočega podniza}
    \begin{block}{Skrčitveno pravilo 3}
        Če obstajata prepisovalni pravili
        \begin{align*}
            A &\rightarrow \alpha_1 \beta \alpha_2, \\
            B &\rightarrow \beta,
        \end{align*}
        kjer so $\alpha_1, \alpha_2, \beta \in (V \cup \Sigma)^*$, $\abs{\beta} \geq 2$
        in $\alpha_1 \neq \varepsilon \vee \alpha_2 \neq \varepsilon$,
        \begin{itemize}
            \item nadomestimo pravilo $A \rightarrow \alpha_1 \beta \alpha_2$ z
            $A \rightarrow \alpha_1 B \alpha_2$.
        \end{itemize}
    \end{block}
    \note{
        \begin{itemize}
            \item Jezik spremenjene dopsutne gramatike je enak jeziku stare.
            \item Nova dopustna gramatika je ``bližje'' zadostitvi $1.$ zahtevi 
            definicije neskrčljive gramatike:

            Vsak $A \in V$, $A \neq S$ nastopa vsaj dvakrat v desni strani prepisovalnih pravil.

            \item Vprašamo se ali z metodo najdaljšega ujemajočega podniza pridelamo neskrčljivo
            gramatiko v končno mnogo korakih?
        \end{itemize}
    }
\end{frame}

\begin{frame}
    \frametitle{Metoda najdaljšega ujemajočega podniza}
    \begin{definicija}
        Za dopustno gramatiko $G$ je $C(G) = 2 \abs{G} - \abs{V}$.
    \end{definicija}
    \pause
    \begin{trditev}
        Za vsako dopustno gramatiko $G$ je $C(G) > 0$.
    \end{trditev}
    \begin{trditev}
        Naj bo $G$ dopustna gramatika in $G^\prime$ dopustna gramatika dobljena z uporabno nekega
        skrčitvenega pravila na $G$. Potem je $C(G^\prime) < C(G)$. 
    \end{trditev}
    \pause
    \begin{izrek}
        Iz dopustne gramatike $G$ dobimo neskrčljivo gramatiko $G^\prime$ z uporabo največ 
        $C(G) - 1$ skrčitvenih previl.
    \end{izrek}
\end{frame}

\section{Binarno kodiranje gramatike}

\begin{frame}
    \frametitle{Binarno kodiranje gramatike}
    \tikzset{block/.style={draw, thick, text width=3cm, minimum height=1.5cm, align=center}, line/.style={-latex}}
    \begin{figure}[H]
        \centering
        \scalebox{0.8}{
        \begin{tikzpicture}
            \node (a) {$w$};
            \node[block,right=of a, xshift=6mm] (b) {prirejanje gramatike};
            \node[right=of b, xshift=3mm] (c) {$G_w$};
            \node[block,right=of c, xshift=3mm] (d) {kodiranje gramatike};
            \node[right=of d, xshift=6mm] (e) {$B(G_w)$};

            \draw[line] (a)-- (b);
            \draw[line] (b)-- (c);
            \draw[line] (c)-- (d);
            \draw[line] (d)-- (e);
        \end{tikzpicture}
        }
    \end{figure}
    \begin{definicija}
        \emph{Binarno kodiranje dopustne gramatike} je preslikava 
        \begin{align*}
            B \colon \G(\A) &\to \{ 0, 1 \}^+, \\
            G &\mapsto B(G).
        \end{align*}
    \end{definicija} 
    \note{
        \begin{itemize}
            \item Sedaj poznamo dva razreda prirejanj gramatike $G_w$.
        \end{itemize}
    }
\end{frame}

\begin{frame}
    \begin{izrek}
        Obstaja bijektivno binarno kodiranje dopustne gramatike, da
        \begin{enumerate}
            \item $ \forall G_1, G_2 \in \G(\A), G_1 \neq G_2 $ niz $B(G_1)$ ni predpona niza $B(G_2)$,
            \item $ \forall G \in \G(\A) \colon \abs{B(G)} \leq \abs{\A} + 4 \abs{G} + \lceil H(G) \rceil$.
        \end{enumerate}
    \end{izrek}
\end{frame}

\begin{frame}
    \begin{definicija}
        \emph{Stiskanje niza abecede $\A$ z gramatikami $\G(\A)$} je 
        par preslikav kodne in dekodne preslikave $\varPhi = (\kappa, \delta)$. Kodna preslikava je
        \begin{align*}
            \kappa \colon A^+ &\to \{ 0, 1\}^+,\\
            w &\mapsto B(\pi(w)),
        \end{align*}
        kjer je $\pi$ prirejanje gramatike nizu abecede $\A$ in
        $B$ binarno kodiranje dopustne gramatike.
    \end{definicija}
    \note{
        \begin{itemize}
            \item Odvečnost meri količino ponavljajočih se ali predvidljivih podatkov znotraj
            sporočila, ki jih je mogoče odstraniti, da se prihrani prostor, brez izgube informacije.
            \item Odvečnost stiskanja z asimptotsko kompaktnim prirejanjem konvergira proti $0$ v 
            odvisnosti od izbire kodiranja znotraj razreda.
            \item Stiskanje z neskrčljivim prirejanjem tudi stiskanje z asimptotsko kompaktnim 
            prirejanjem in odvečnost konvergira enakomerno proti $0$ za vsa stiskanja z neskrčljivim 
            prirejanjem, vsaj tako hitro kot $\frac{\log_2 \log_2(n)}{\log_2(n)}$ pomnoženo z neko konstanto.
        \end{itemize}
    }
\end{frame}


\end{document}