\documentclass{beamer}

\usepackage[T1]{fontenc}
\usepackage[utf8]{inputenc}
\usepackage[slovene]{babel}
\usepackage{palatino}
\usefonttheme{serif}
\usepackage{amsmath,amssymb,amsfonts, mathtools}
\graphicspath{{./images/}}
\setbeamertemplate{enumerate items}[default]

\usetheme{CambridgeUS}
\usecolortheme{dolphin}
\setbeamertemplate{navigation symbols}{}

\linespread{1.2}

\newcommand{\N}{\mathbb{N}}
\newcommand{\Z}{\mathbb{Z}}
\newcommand{\Q}{\mathbb{Q}}
\newcommand{\R}{\mathbb{R}}
\newcommand{\C}{\mathbb{C}}
\newcommand{\G}{\mathcal{G}}
\newcommand{\A}{\mathcal{A}}

\theoremstyle{definition}
\newtheorem{definicija}{Definicija}[section]
\newtheorem{izrek}[definicija]{Izrek}
\newtheorem{trditev}[definicija]{Trditev}
\newtheorem{posledica}[definicija]{Posledica}
\newtheorem{opomba}[definicija]{Opomba}


\newcommand\Vtextvisiblespace[1][.3em]{%

\mbox{\kern.06em\vrule height.3ex}%
\vbox{\hrule width#1}%
\hbox{\vrule height.3ex}}

\title[Gramatike za kodiranje podatkov]{Kontekstno-neodvisne gramatike \\ za kodiranje in stiskanje podatkov}
\author{Janez Podlogar}
\institute[UL-FMF]{Univerza v Ljubljani, Fakulteta za matematiko in fiziko}
\date[Maj 2023]{5. 5. 2023}

\begin{document}

\begin{frame}
    \titlepage
\end{frame}

\section{Ponovitev}

\begin{frame}{Spomnimo se}
    S kontekstno-neodvisnih gramatik želimo stisniti poljuben niz.
    \pause
    \begin{definicija}
        \begin{itemize}
            \item \textit{Abeceda} je končna neprazna množica $ \Sigma $,
            \item \textit{Množico vseh končnih nizov abecede} $ \Sigma $ označimo s $ \Sigma^* $,
            \item \textit{Dolžina niza} $ w $ je število znakov abecede v $ w $, označimo z $ |w| $.
        \end{itemize}
    \end{definicija}
    \pause
    \begin{exampleblock}{Primer nizov abecede}<2->
        Naj bo $ \Sigma = \{ a,b,c \} $ abeceda, potem sta niza
        \[ 
            ab \in \Sigma^* , \quad cababcccababcccab \in \Sigma^*
        \]
    \end{exampleblock}
\end{frame}

\begin{frame}{Kontekstno-neodvisne gramatike}
    \begin{definicija}
        \textit{Kontekstno-neodvisna gramatika} je četverica $ G = ( V, \Sigma, P, S ) $, kjer je
        \begin{itemize}
            \item<2-> $ V $ končna množica \textit{nekončnih simbolov},
            \item<3-> abeceda $ \Sigma $ množica \textit{končnih simbolov},
            \item<4-> $ P \subseteq V \times ( V \cup \Sigma )^* $ celovita relacija,
            \item<5-> $ S \in V $ \textit{začetni simbol}.
        \end{itemize}
    \end{definicija}
    \begin{definicija}<6->
        \textit{Jezik kontekstno-neodvisne gramatike $ G $} je množica vseh nizov, ki jih lahko izpeljemo
        z gramatiko $ G $, označimo ga z $ L(G) $.
    \end{definicija}
\end{frame}

\begin{frame}{Stiskanje niza s kontekstno-neodvisno gramatiko}
    \begin{exampleblock}{Stiskanje niza $ w = \mathit{cababcccababcccab} $}
        Naj bo $ G_w = ( V, \Sigma, P, S ) $, kjer je 
        \begin{itemize}
            \item<2-> $ V = \{ S, A, B, C \} $
            \item<3-> $ \Sigma = \{ a, b, c \} $
            \item<4-> $ P = \{ S  \rightarrow  \mathit{cCCA}, A  \rightarrow  \mathit{ab}, B  
            \rightarrow  \mathit{ccc}, C  \rightarrow  \mathit{AAB} \} $
            \item<5-> $ S = S $
        \end{itemize}
        \pause
        \onslide<6->{$ S = cCCA $}
        \pause
        \onslide<7->{$ \xRightarrow{C} cAABAAB $}
        \pause
        \onslide<8->{$ \xRightarrow{B} cAAcccAAccc $}
        \pause
        \onslide<9->{$ \xRightarrow{A} cababcccababccc $}
        \onslide<10->{$ = w $}
        \pause \\
        \onslide<11->{Velja, da je 
        \[
            L(G_w) = \{ w \}.
        \]
        }
    \end{exampleblock}
\end{frame}

\section{Dopustne gramatike}

\begin{frame}{Deterministična kontekstno-neodvisna gramatika}
    \begin{definicija}
        Kontekstno-neodvisna gramatika $G$ je \textit{deterministična}, če vsak nekončen simbol $ A \in V $,
        nastopa natanko enkrat kot levi član nekega produkcijskega pravila.
        Kontekstno-neodvisna gramatika, ki ni deterministična, je \textit{nedeterministična}.
    \end{definicija}
    \pause
    \begin{trditev}
        Naj bo $G$ deterministična kontekstno-neodvisna gramatika. Potem je jezik gramatike $G$ enojec ali pa prazna množica.
    \end{trditev}
\end{frame}

\begin{frame}{Deterministična kontekstno-neodvisna gramatika}
    \begin{exampleblock}{Deterministična kontekstno-neodvisna gramatika s praznim jezikom}
        Naj bo $ G = ( V, \Sigma, P, S ) $, kjer je 
        \begin{itemize}
            \item<2-> $ V = \{ S \} $
            \item<3-> $ \Sigma = \{ a \} $
            \item<4-> $P = \{ S \rightarrow S \} $
            \item<5-> $ S = S $
        \end{itemize}
        \pause
        \onslide<6->{Gramatika je očitno deterministična in velja $ L(G) = \emptyset $.
        }
    \end{exampleblock}
\end{frame}

\begin{frame}{Deterministična kontekstno-neodvisna gramatika}
    \begin{exampleblock}{Deterministična kontekstno-neodvisna gramatika z neuporabnimi simboli}
        Naj bo $ G = ( V, \Sigma, P, S ) $, kjer je 
        \begin{itemize}
            \item $ V = \{ S, A \} $
            \item $ \Sigma = \{ a \} $
            \item $P = \{ S \rightarrow a, A \rightarrow a \} $
            \item $ S = S $
        \end{itemize}
        \pause
        \onslide<2->{Gramatika je očitno deterministična in velja $ L(G) = \{ a \} $.
        }
    \end{exampleblock}
\end{frame}

\begin{frame}{Neuporabni simboli in prazen jezik}
    \begin{definicija}
        Pravimo, da kontekstno-neodvisna gramatika $G$ \textit{ne vsebuje neuporabnih simbolov}, ko za vsak simbol
        $ y \in V \cup \Sigma, \ y \neq S $, obstaja končno mnogo nizov $ \alpha_1, \alpha_2, \ldots, \alpha_n \in 
        (V \cup \Sigma)^* $ tako, da se $y$ pojavi v vsaj enem izmed teh nizov in velja
        \[
            S = \alpha_1 \Rightarrow \alpha_2 \Rightarrow \cdots \alpha_n \in L(G).
        \] 
    \end{definicija}
\end{frame}

\begin{frame}{Dopustne gramatike}
    \begin{definicija}
        Kontekstno-neodvisna gramatika $G$ je \textit{dopustna}, če je
        \begin{itemize}
            \item Deterministična,
            \item Ne vsebuje neuporabnih simbolov,
            \item  $ L(G) \neq \emptyset $ in prazen niz ne nastopa kot desni član kateregakoli produkcijska pravila v $ P $.
        \end{itemize}
    \end{definicija}
    \pause
    \begin{posledica}
        Jezik dopustne kontekstno-neodvisne gramatike je enojec.
    \end{posledica}
\end{frame}

\begin{frame}
    \begin{exampleblock}{Za dopustno gramatiko je dovolj podati produkcijska pravila}
        Podana imamo produkcijska pravila 
        \begin{itemize}
            \item $ P = \{ A_0 \rightarrow aA_1A_2A_3, A_1 \rightarrow ab, A_2 \rightarrow A_1b, A_3 \rightarrow A_2b \} $ 
        \end{itemize}
        \pause
        \onslide<1->{Nekončni simboli gramatike so 
        \[
             V = \{ A_0, A_1, A_2, A_3 \}.
        \] 
        }
        \pause
        \onslide<2->{
        Končni simboli gramatike so
        \[
            \Sigma = \{ a, b \}.
        \] 
        }
        \pause
        \onslide<3->{Začetni simbol je
        \[
            S = A_0.
        \]
        }
    \end{exampleblock}
\end{frame}

\section{D0L sistem}

\begin{frame}{Endomorfizem}
    \begin{definicija}
        Naj bo $ \Sigma $ abeceda. \textit{Endomorfizem množice $ \Sigma^* $} je preslikava $ f \colon \Sigma^* \to \Sigma^* $
        tako, da je
        \[
            f(\varepsilon) = \varepsilon \text{ in } \forall w,u \in \Sigma^* \colon f(wu) = f(w)f(u).
        \]
        \pause
        Induktivno definiramo
        \begin{align*}
            f^0(w) &= w, \\
            f^1(w) &= f(w), \\
            f^k(w) &= f(f^{k-1}(w)),
        \end{align*}
        kjer je $ w \in \Sigma^* $ in $ k \geq 2 $ celo število.
    \end{definicija}
\end{frame}

\begin{frame}{D0L sistem}
    \begin{definicija}
        \textit{D0L sistem} je trojica $ D = (\Sigma, \ f, \ w) $, kjer je
        \begin{itemize}
            \item<2-> $ \Sigma $ abeceda,
            \item<3-> $ f $ endomorfizem množice $ \Sigma^* $,
            \item<4-> $ w \in \Sigma^* $ \textit{aksiom}.
        \end{itemize}
        \pause

        \onslide<5->{Sistem tvori zaporedje nizov $ \{ f^k(w) \mid k = 0, 1,2 \ldots \} $, 
        ki ima \textit{fiksno točko $ w^* $}, če velja
        \begin{gather*}
            w^* \in \{ f^k(w) \mid k = 0, 1,2 \ldots \}, \\
            f(w^*)= w^*.
        \end{gather*}
        }
    \end{definicija}
\end{frame}

\begin{frame}{D0L sistem prirejen gramatiki}
    \begin{definicija}
        Naj bo G deterministična kontekstno-neodvisna gramatika v kateri prazen niz ne nastopa kot desni član kateregakoli
        produkcijska pravila. Na $ (V \cup \Sigma)^* $ definiramo endomorfizem $ f_G $ tako, da 
        \begin{gather*}
            \forall a \in \Sigma \colon f_G(a) = a; \\
            \text{če je } A \rightarrow \alpha \text{ produkcijsko pravilo, potem je } f_G(A) = \alpha.
        \end{gather*}
        \pause
        D0L sistem $ (V \cup \Sigma, \ f_G, \ S) $ imenujemo \textit{D0L sistem prirejen gramatiki $G$}.
    \end{definicija}
\end{frame}

\begin{frame}{D0L sistem prirejen gramatiki}
    \begin{izrek}
        Naj bo $G$ dopustna kontekstno-neodvisna gramatika.
        Potem jezik gramatike $G$ ustreza fiksni točki D0L sistema $ (V \cup \Sigma, \ f_G, \ S) $.
    \end{izrek}
    \pause
    \begin{posledica}
        Jezik dopustne kontekstno-neodvisna gramatike $G$ je
        \[
            L(G) = \{f_G^{|V|}(S) \}.
        \]
    \end{posledica}
\end{frame}

\begin{frame}{Karakterizacija dopustnih gramatik preko D0L sistema}
    \begin{izrek}
        Naj bo $G$ kontekstno-neodvisna gramatika v kateri prazen niz ne nastopa kot desni član kateregakoli
        produkcijska pravila. Gramatika $G$ je dopustna natanko takrat, ko je $ f_G^{|V|}(S) \in \Sigma^+ $
        in vsak simbol iz $ V \cup \Sigma $ nastopa v vsaj enem izmed nizov $ f_G^i(S) \text{, kjer je } i = 0, 1, \ldots, |V| $.
    \end{izrek}
\end{frame}

\begin{frame}
    \begin{exampleblock}{Preverimo, da je gramatika dopustna in izračunajmo jezik}
        Podana imamo produkcijska pravila 
        \begin{itemize}
            \item<1-> $ P = \{ A_0 \rightarrow aA_1A_2A_3, A_1 \rightarrow ab, A_2 \rightarrow A_1b, A_3 \rightarrow A_2b \} $
            \item<2-> $ V = \{ A_0, A_1, A_2, A_3 \} $ in $ \Sigma = \{ a,b \} $
        \end{itemize}
        \pause
        \begin{align*}
            \uncover<3->{f_G(A_0) &= aA_1A_2A_3 \\}
            \uncover<4->{f_G^2(A_0) &= aabA_1bA_2b \\}
            \uncover<5->{f_G^3(A_0) &= aababbA_1bb \\}
            \uncover<5->{f_G^4(A_0) &= aababbabbb }
        \end{align*}
        \pause
        \onslide<6->{Vsak od simbolov $A_1, A_2, A_3, a, b$ se pojavi vsaj enkrat v zgoraj izračunanih nizih
        }
        \onslide<7->{in $ f_G^4(A_0) \in \Sigma^+ $.}
    \end{exampleblock}
\end{frame}

\section{Pretvorba niza v gramatiko}

\begin{frame}{Pretvorba niza v gramatiko}
    Z $\A$ označimo poljubno abecedo z vsaj dvema črkama
    in fiksiramo števno neskončno množico simbolov
    \[
        \{A_0, A_1, A_2, \ldots \},
    \]
    Predpostavimo, da nobeden
    od simbolov $ A_0, A_1, A_2, \ldots $ ne nastopa v abecedi $\A$.
\end{frame}

\begin{frame}{Pretvorba niza v gramatiko}
    \begin{definicija}
        Naj bo $\A$ abeceda. Z $ \G(\A) $ označimo podmnožico vseh kontekstno neodvisnih gramatik
        $G$, ki izpolnjujejo naslednje pogoje:
        \begin{enumerate}
            \item<1-> $G$ je dopustna;
            \item<2-> $ \Sigma(G) \subseteq \A $;
            \item<3->$ S(G) = A_0 $;
            \item<4-> $ V(G) = \{ A_0, A_1, A_2, \ldots, A_{|V(G)| - 1} \} $;
            \item<5-> Če naštejemo nekončne simbole v vrstnem redu pojavitve v nizu
            \[
            f_G^0(A_0) f_G^1(A_0) f_G^2(A_0) \dots f_G^{|V(G)| - 1}(A_0),
            \]
            dobimo urejen seznam $ A_0, A_1, A_2, \ldots, A_{|V(G)| - 1} $.
        \end{enumerate}
    \end{definicija}
\end{frame}

\begin{frame}{Pretvorba niza v gramatiko}
    \begin{opomba}
         Naj bo $ G \in \G(\A) $ in sta izpolnjena prva dva pogoja zgornje definicije.
         Potem lahko enolično preimenujemo spremenljivke iz $ V(G) $ tako, da dobimo gramatiko, ki izpolnjuje tudi ostale tri pogoje. Pridelano gramatiko označimo z $[G]$ in očitno velja $ [G] \in \G(\A) $.
    \end{opomba}
\end{frame}

\begin{frame}{Pretvorba niza v gramatiko}
    \begin{definicija}
        Naj bo $\A$ abeceda in $G$ kontekstno-neodvisna gramatika. 
        \textit{Pretvorba niza v gramatiko} je preslikava
        \begin{align*}
            \A^+ &\rightarrow \G(\A), \\
            w &\mapsto G_w.
        \end{align*}
        \pause
        Z $|G|$ označimo skupno število vseh desnih članov produkcijskih pravil gramatike vključno s ponovitvami. Pretvorba niza v gramatiko je \textit{asimptotsko kompaktna}, če je
        \[
            \lim_{n \rightarrow \infty} \max_{w \in \A^n} \frac{|G_w|}{|w|} = 0.
        \]
    \end{definicija}
\end{frame}

% Ali ni to kar podpoglavje od preobrazbe?
\subsection{Biskecijska pretvorba niza v gramatiko}

\begin{frame}{Biskecijska pretvorba niza v gramatiko}
    \begin{definicija}
        Naj bo $ w = u_1 u_2 \cdots u_n \in \A^+ $.
        Vpeljemo \textit{bisekcijsko členitev niza}
        \[
            B(w) = \{ w \} \cup \{ u_i \cdots u_j \mid \tfrac{i-1}{j-i+1} \text{ in } \log_2{(j-i+1)} \text{ sta celi števili}  \}.
        \]
        \pause
        Za vsak podniz $ u \in B(w) $ z $ ( b(u, L), b(u,R) )$ označimo členitev podniza, kjer sta niza $ b(u, L) $ in $ b(u,R) $ enake dolžine.
    \end{definicija}
\end{frame}

\begin{frame}{Biskecijska pretvorba niza v gramatiko}
    \begin{exampleblock}{Biskecijska členitev niza}
        Naj bo $w = 0001010 $, pote je
        \[
            B(w) = \{ w, 0001, 01, 0, 00, 1 \}.
        \]
    \end{exampleblock}
\end{frame}

\begin{frame}{}
    \begin{definicija}
        \textit{Bisekcijska pretvorba niza v gramatiko} je pretvorba niza v gramatiko $ w \mapsto [G_{w}] $, kjer je
        \begin{itemize}
            \item<2-> $ V(G_{w}) = \{ A^u \mid u \in B(w) \} $,
            \item<3-> $ \Sigma(G_{w}) = \{ u \in B(w) \mid |u| = 1 \} $,
            \item<4-> $ S(G_{w}) = A^{w} $.
        \end{itemize}
    \end{definicija}
\end{frame}

\begin{frame}{}
    \begin{definicija}
        Množica $ P(G_{w}) $ vsebuje sledeča produkcijska pravila
        \begin{itemize}
            \item<1-> Če je $ |u| = 1 $, je produkcijsko pravilo oblike
            \[
                A^u \rightarrow u.  
            \]
            \item<2-> Če je $ \log_2{|u|} $ celo število, je produkcijsko pravilo oblike
            \[
                A^u \rightarrow A^{b(u,L)} A^{b(u,R)}. 
            \]
            \item<3-> Če $ \log_2{|u|} $ ni celo število, torej je $ u = w $, je produkcijsko pravilo oblike
            \[
                A^u \rightarrow A^{u_1} A^{u_2} \cdots A^{u_t},
            \]
            kjer je $ (u_1, u_2, \ldots, u_t) $ enolična členitev niza $u$, da je $ \forall i \colon u_i \in B(w) $ in $ |u| > |u_1| > |u_2| > \cdots > |u_t| $.
        \end{itemize}
    \end{definicija}
\end{frame}

\begin{frame}
    \begin{exampleblock}{Biskecijska pretvorba niza $ w = 0001010 $ v gramatiko }
        Vemo, da je $ B(w) = \{ w, 0001, 01, 0, 00, 1 \} $
        \pause
        in pravila $ G_{w} $ so
        \begin{align*}
            \uncover<3->{A^{w} &\rightarrow A^{0001} A^{01} A^{0} \\}
            \uncover<4->{A^{0001} &\rightarrow A^{00} A^{01} \\}
            \uncover<5->{A^{00} &\rightarrow A^0 A^0 \\}
            \uncover<6->{A^{01} &\rightarrow A^0 A^1 \\}
            \uncover<7->{A^0 &\rightarrow 0 \\}
            \uncover<8->{A^1 &\rightarrow 1 }
        \end{align*}
    \end{exampleblock}
\end{frame}

\begin{frame}
    \begin{exampleblock}{Biskecijska pretvorba niza $ w = 0001010 $ v gramatiko }
        Vemo, da je $ B(w) = \{ w, 0001, 01, 0, 00, 1 \} $
        in pravila $ G_{w} $ so
        \begin{align}
            A^{w} &\rightarrow A^{0001} A^{01} A^{0} \\
            A^{0001} &\rightarrow A^{00} A^{01} \\
            A^{00} &\rightarrow A^0 A^0 \\
            A^{01} &\rightarrow A^0 A^1 \\
            A^0 &\rightarrow 0 \\
            A^1 &\rightarrow 1
        \end{align}
    \end{exampleblock}
\end{frame}

\begin{frame}
    \begin{exampleblock}{Biskecijska pretvorba niza v gramatiko w = 0001010}
        Vemo, da je $ B(w) = \{ w, 0001, 01, 0, 00, 1 \} $
        in pravila $ [G_{w}] $ so
        \begin{align*}
            A_0 &\rightarrow A_1 A_2 A_3 \tag{1} \\
            A_1 &\rightarrow A_4 A_2 \tag{2}\\
            A_2 &\rightarrow A_3 A_5 \tag{4}\\
            A_3 &\rightarrow 0 \tag{5} \\
            A_4 &\rightarrow A_3 A_3 \tag{3} \\
            A_5 &\rightarrow 1 \tag{6}
        \end{align*}
    \end{exampleblock}
\end{frame}

\section{Diplomska naloga}

\begin{frame}{Načrt nadaljnjega dela}
    \begin{itemize}
        \item Asimptotsko kompaktne pretvorbe niza in pripadajoči rezultati,
        \item Lempel–Ziv pretvorba niza in biskecijska pretvorba niza,
        \item Implementacija in empirična analiza zgoraj navednih pretvorb.
    \end{itemize}
\end{frame}

\end{document}